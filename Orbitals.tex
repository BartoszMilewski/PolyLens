\documentclass[11pt]{amsart}          
\usepackage{amsfonts}
\usepackage{amssymb}  
\usepackage{amsthm} 
\usepackage{amsmath} 
\usepackage{tikz-cd}
\usepackage{float}
\usepackage[]{hyperref}
\usepackage{minted}
\hypersetup{
  colorlinks,
  linkcolor=blue,
  citecolor=blue,
  urlcolor=blue}
\newcommand{\hask}[1]{\mintinline{Haskell}{#1}}
\newenvironment{haskell}
  {\VerbatimEnvironment
  	\begin{minted}[escapeinside=??, mathescape=true,frame=single, framesep=5pt, tabsize=1]{Haskell}}
  {\end{minted}}
\newcommand*\circled[3]
  {\tikz[baseline={([yshift=#2]current bounding box.center)}]{
      \node[shape=circle,draw,inner sep=#1] (char) {#3};}}
\newcommand{\actL}{\mathbin{\circled{0.5pt}{-0.55ex}{\scalebox{0.75}{\tiny{$\mathsf{L}$}}}}}
\newcommand{\actR}{\mathbin{\circled{0.5pt}{-0.55ex}{\scalebox{0.75}{\tiny{$\mathsf{R}$}}}}}

\newcommand{\cat}[1]{\mathcal{#1}}%a generic category
\newcommand{\Cat}[1]{\mathbf{#1}}%a named category
\newcommand{\Set}{\Cat{Set}}

\author{Bartosz Milewski}
\title{Orbital Functors}

\begin{document}
\maketitle{}

\section{Category of Orbital Functors}

Action of a monoidal category $\cat M$ on $\cat C$ given by an endofunctor:
\[ m \bullet \colon \cat C \to \cat C \]
satisfying the laws:
\begin{align*}
1 \bullet a &= a \\
(m \cdot n) \bullet a &= m \bullet (n \bullet a)
\end{align*}

Lax orbital functor $F \colon \cat C \to \Set$ is equipped with the structure:
\[ \alpha_{m, a} \colon F a \to F (m \bullet a) \]
which is a natural transformations that let's us extend the functor along the orbits of $m \bullet$. We have the following coherence conditions:
\begin{align*}
\alpha_{1, a} &= id_{F a} \\
\alpha_{n \cdot m, a} &= \alpha_{n, a} \circ \alpha_{m, a}
\end{align*}

These functors form a category $\mathbf{Orb}$.  A morphisms from $(F, \alpha)$ to $(G, \beta)$ is a natural transformation that maps the two structures:
\[
\begin{tikzcd}
F a
\arrow[r, "\mu_a"]
\arrow[d, "\alpha"]
& G a
\arrow[d, "\beta"]
\\ F (m \bullet a) 
\arrow[r, "\mu_{m \bullet a}"]
& G (m \bullet a)
\end{tikzcd}
\]

\section{Category of Orbits}

Category $\cat O$ of orbits has the same objects as $\cat C$, but its hom-sets that are given by the colimit formula. We use the coend notation for the colimit (a colimit of $F$ is a coend of a profunctor $P\langle x, y \rangle = F y$):
\[ \cat O(a, b) = \int^m \cat C(a, m \bullet b) \]
An element of the hom-set can be constructed by injecting a pair $(m, f \colon a \to m \bullet b)$ into the colimit.

Composition: 
\[ \circ \colon \cat O(b, c) \times \cat O(a, b) \to \cat O(a, c) \]
is a member of the set:

\[ \Set \big(\int^m \cat C(b, m \bullet c) \times \int^{n} \cat C(a, n \bullet b) , \int^k \cat C(a, k \bullet c) \big) \]
By cocontuity of the hom-functor, we can replace the colimit with the limit (using the end notation):
\[\int_{m,n}  \Set \big(\cat C(b, m \bullet c) \times \cat C(a, n \bullet b) , \int^n \cat C(a, n \bullet c) \big) \]
Given two morphisms $f \colon b \to m \bullet c $ and $g \colon a \to n \bullet b$, we can produce the composition (using the functoriality of $m \bullet$):
\[ (n \bullet f ) \circ g \colon a \to (n \cdot m) \bullet c \]
and inject it into the colimit with $k = n \cdot m$.
The identity morphism at $a$ is given by the pair $(1, id_a)$

\section{Presheaves on Orbits}

The category of presheaves $[ \cat O^{op}, \Set]$ is isomorphic to the category $\mathbf {Orb}$ of lax orbital functors. 


\[
\begin{tikzcd}
\Set
& \Set
\\ \cat O^{op}
\arrow[u, "\hat F"]
& \cat C
\arrow[u, "{(F, \alpha)}"]
\end{tikzcd}
\]

Indeed, given a presheaf $\hat F: \cat O^{op} \to \Set$ we construct a functor $F \colon \cat C \to \Set$ which on objects is the same as $\hat F$. 

Let's define the action of $F$ on morphisms. Given a morphism $f \colon a \to b$ in $\cat C$, we first construct a pair $(f, 1)$ with the unit monoidal action, inject it into the colimit to get an element of $\cat O (a, b)$, and apply $\hat F$ to it.

The resulting functor is automatically lax orbital, with the structure $\alpha_{m, a}$ given by the action of $\hat F$ on the colimit constructed from $(id_{m \bullet a}, m)$, which is a member of $\cat O(m \bullet a, a)$ 

\[
\begin{tikzcd}
\Set
& \hat F a
\arrow[r, "{\alpha_{m, a}}"]
& \hat F (m \bullet a)
\\ \cat O^{op}
\arrow[u, "\hat F"]
& a
& m \bullet a
\arrow[l, "{\cat O(m \bullet a, a)}"]
\end{tikzcd}
\]


\[ \alpha_{m, a} \in \hat F \big(\cat O (m \bullet a, a)\big) = \hat F\big(\int^m \cat C (m \bullet a, m \bullet a) \big)\]

Conversely, given a lax orbital functor $F \colon \cat C \to \Set$, we can build a presheaf $\hat F \colon \cat O^{op} \to \Set$ that is identity on objects. Its action on morphisms is a mapping of hom-sets:
\[ \cat O (b, a) \to \Set (F a, F b) \]
Such a mapping is a member of:
\[ \Set \big (\int^m \cat C (b, m \bullet a), \Set (F a, F b) \big) \]
Using continuity and the currying adjunction in $\Set$ we get:
\[ \int_m \Set \big ( \cat C (b, m \bullet a) \times F a, F b \big) \]
Given a morphism $f \colon b \to m \bullet a$ and the set $F a$, we first apply $\alpha_{m, a}$ to $F a$ to get $F (m \bullet a)$ and follow it by $F f$ to get $F b$.

The mapping between these two functor categories is functorial. Consider a natural transformation $\mu \colon \hat F \to \hat G$. For any morphisms $g \in \cat O(b, a)$, the naturality square reads:
\[
\begin{tikzcd}
\hat F b
\arrow[r, "\mu_b"]
& \hat G b
\\
\hat F a
\arrow[u, "\hat F g"]
\arrow[r, "\mu_a"]
& \hat G a
\arrow[u, "\hat G g"]
\end{tikzcd}
\]

We map $\mu$ to a natural transformation between two orbital functors: $(F, \alpha) \to (G, \beta)$. Since $\hat F$ and $F$ are the same on objects, the same $\mu$ can be used to map $F$ to $G$. 

Naturality condition, however, involves the mapping of morphisms. We have to show that, for any $f \colon a \to b$ the following square commutes:
\[
\begin{tikzcd}
F b
\arrow[r, "\mu_b"]
& G b
\\
F a
\arrow[u, "F f"]
\arrow[r, "\mu_a"]
&  G a
\arrow[u, "G f"]
\end{tikzcd}
\]
The lifting of $f$ is done by injecting $(f, 1)$ into $\cat O(a, b)$ and applying $\hat F$ (or $\hat G$) to it. So the naturality of $\mu$ with respect to $F$ follows from the naturality with respect to $\hat F$.

Similarly, the induced structure maps are preserved, since they are given by the lifting of morphisms in $\cat O(m \bullet a, a)$:
\[
\begin{tikzcd}
F a
\arrow[r, "\mu_a"]
\arrow[d, "\alpha"]
& G a
\arrow[d, "\beta"]
\\ F (m \bullet a) 
\arrow[r, "\mu_{m \bullet a}"]
& G (m \bullet a)
\end{tikzcd}
\]

\section{Tannakian Reconstruction}

Tannakian reconstruction for categories expresses a hom-set in a category as an end over all co-presheaves on that category:
\[ \int_{\hat F \colon [\cat C, \Set]} \Set (\hat F a, \hat F b) \cong \cat C(a, b) \]
In particular, if we choose our category as the (opposite of the) category of orbits, we get:
\[ \int_{\hat F \colon [\cat O^{op}, \Set]} \Set (\hat F a, \hat F b) \cong \cat O(b, a) \]
As we've seen, this category is equivalent to the category of orbital functors, so we get:
\[ \int_{ F \colon \mathbf{Orb}} \Set (F a, F b) \cong \int^{m \colon \cat M} \cat C(b, m \bullet a) \]
The left-hand side is a set of natural transformations between two fiber functors. A fiber functor maps a functor $F$ to its value (a set) at a fixed object $a$:
\[ \Phi_a \colon F \mapsto F a \]
Conceptually, a fiber functor ``probes'' the environment around a fixed object. 

The advantage of this representation is that it replaces a somewhat messy composition of morphisms in $\cat O$ (the right hand-side) with simple composition of functions in $\Set$ (the left-hand side).


\end{document}