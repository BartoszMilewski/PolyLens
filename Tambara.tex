\documentclass[11pt]{amsart}          
\usepackage{amsfonts}
\usepackage{amssymb}  
\usepackage{amsthm} 
\usepackage{amsmath} 
\usepackage{tikz-cd}
\usepackage{float}
\usepackage[]{hyperref}
\usepackage{minted}
\hypersetup{
  colorlinks,
  linkcolor=blue,
  citecolor=blue,
  urlcolor=blue}
\newcommand{\hask}[1]{\mintinline{Haskell}{#1}}
\newenvironment{haskell}
  {\VerbatimEnvironment
  	\begin{minted}[escapeinside=??, mathescape=true,frame=single, framesep=5pt, tabsize=1]{Haskell}}
  {\end{minted}}
\newcommand{\cat}[1]{\mathcal{#1}}%a generic category
\newcommand{\Cat}[1]{\mathbf{#1}}%a named category

\author{Bartosz Milewski}
\title{A Category of Generalized Tambara Modules}
\begin{document}
\maketitle{}

\section{Motivation}

\section{Tambara Category}

We start with a monoidal category $\cat M$ and its action in a pair of categories $\cat C$ and $ \cat D$ (we use the same infix symbol for both):
\begin{align*} 
\bullet &\colon \cat M \times \cat C \to \cat C 
\\
 \bullet &\colon \cat M \times \cat D \to \cat D 
\end{align*}
A Tambara module is a profunctor $p \colon \cat C^{op} \times \cat D \to \Cat{Set}$ equipped with a family of transformations:
\[ \alpha_{m, \langle a, b \rangle} \colon p \langle a, b \rangle \to p \langle m \bullet a, m \bullet b \rangle \]
natural in $m$ and dinatural in $\langle a, b \rangle$, satisfying the unit and associativity conditions. 

Tambara modules form a category.  A morphism between Tambara modules $(p, \alpha)$ and $(q, \alpha')$ is a natural transformation:

\[ \rho_{\langle a, b \rangle} \colon p \langle a, b \rangle \to q \langle a, b \rangle \]
 satisfying the coherence conditions:
\[
 \begin{tikzcd}
 p \langle a, b \rangle
 \arrow[r, "{\alpha_{m, \langle a, b \rangle}}"]
 \arrow[dd, "{\rho_{ \langle a, b \rangle}}"']
& p \langle m \bullet a, m \bullet b \rangle
 \arrow[dd, "{\rho_{\langle m \bullet a, m \bullet b \rangle}}"]
 \\
  \\
 q \langle a, b \rangle
 \arrow[r, "{\alpha'_{m, \langle a, b \rangle}}"]
& q \langle m \bullet a, m \bullet b \rangle
 \end{tikzcd}
\]
In what follows I'll consider, for the sake of simplicity, Tambara endo-modules of the type  $p \colon \cat C^{op} \times \cat C \to \Cat{Set}$

\section{Generalized Tambara Category}

We start with two monoidal categories $\cat M$ and $\cat N$ with two separate actions on two categories:
\begin{align*} 
\bullet &\colon \cat M \times \cat C \to \cat C 
\\
 \bullet &\colon \cat N \times \cat D \to \cat D 
\end{align*}
The two monoidal actions can also be seen, after currying, as strict monoidal functors:
\begin{align*}
&\cat M \to [\cat C, \cat C]
\\
&\cat N \to [\cat D, \cat D]
\end{align*}

Next we define a category of monoidal functors $\cat M \to \cat N$. Given a functor $c$, we'll write its action on $m \colon \cat M$ as $c \cdot m$. This produces an object of $\cat N$. This action satisfies the usual associativity and unit conditions of a monoidal functor. 

We lift this action to a functor from $\cat C$ to  $\cat D$, or the action:
\[ \bullet \colon \cat M \times \cat C \to \cat D \]
satisfying the coherence conditions:
\[
 \begin{tikzcd}
 a
 \arrow [r, "m \bullet"] 
 \arrow [d, "(c \cdot m) \bullet"']
 & m \bullet a
 \arrow [d, "c \bullet"]
 \\
 (c \cdot m) \bullet a
 \arrow [r, "\cong"]
 & c \bullet (m \bullet a)
 \end{tikzcd}
\]

\[
 \begin{tikzcd}
 a
 \arrow [r, "c \bullet"] 
 \arrow [d, "(n \cdot c) \bullet"']
 & c \bullet a
 \arrow [d, "n \bullet"]
 \\
 (n \cdot c) \bullet a
 \arrow [r, "\cong"]
 & n \bullet (c \bullet a)
 \end{tikzcd}
\]


We have a Tambara module $p$ in $\cat C^{op} \times \cat C$ with the structure:
\[ \alpha_{m, \langle a, b \rangle} \colon p \langle a, b \rangle \to p \langle m \bullet a, m \bullet b \rangle \]
and another one $q$ in $\cat D^{op} \times \cat D$  with the structure:
\[ \beta_{n, \langle s, t \rangle} \colon q \langle s, t \rangle \to q \langle n \bullet s, n \bullet t \rangle \]

A morphism between Tambara modules is a functor $c \colon \cat C \to \cat D$ and a family of functions:
\[ \rho_{c, \langle a, b \rangle} \colon p \langle a, b \rangle \to q \langle c \bullet a, c \bullet b \rangle \]
satisfying the following coherence conditions:
\[
 \begin{tikzcd}
 p \langle a, b \rangle
 \arrow[r, "{\alpha_{m, \langle a, b \rangle}}"]
 \arrow[dd, "{\rho_{c \cdot m, \langle a, b \rangle}}"']
& p \langle m \bullet a, m \bullet b \rangle
 \arrow[dd, "{\rho_{c, \langle m \bullet a, m \bullet b \rangle}}"]
 \\
  \\
 q \langle (c \cdot m)  \bullet a, (c \cdot m) \bullet b \rangle
 \arrow[r, ""]
& q \langle c \bullet (m \bullet a), c \bullet (m \bullet b) \rangle
 \end{tikzcd}
\]

\[
 \begin{tikzcd}
 p \langle a, b \rangle
 \arrow[r, "{\rho_{c, \langle a, b \rangle}}"]
 \arrow[dd, "{\rho_{n \cdot c, \langle a, b \rangle}}"']
& q \langle c \bullet a, c \bullet b \rangle
 \arrow[dd, "{\beta_{n, \langle c \bullet a, c \bullet b \rangle}}"]
 \\
  \\
 q \langle (n \cdot c)  \bullet a, (n \cdot c) \bullet b \rangle
 \arrow[r, ""]
& q \langle n \bullet (c \bullet a), n \bullet (c \bullet b) \rangle
 \end{tikzcd}
\]

This extends the usual definition of a Tambara category which only works over one particular monoidal category. Here we can move between Tambara modules that are defined over different monoidal categories.

\section{References}

\begin{itemize}
\item D. Tambara, \emph{Distributors on a tensor category}. Hokkaido Math. J. 35(2): 379-425 (May 2006)

\item Craig Pastro and Ross Street. \emph{Doubles for monoidal categories}. Theory and
applications of categories, 21(4):61–75, 200
\end{itemize}



\end{document}