\documentclass[11pt]{amsart}          
\usepackage{amsfonts}
\usepackage{amssymb}  
\usepackage{amsthm} 
\usepackage{amsmath} 
\usepackage{tikz-cd}
\usepackage{float}
\usepackage[]{hyperref}
\usepackage{minted}
\hypersetup{
  colorlinks,
  linkcolor=blue,
  citecolor=blue,
  urlcolor=blue}
\newcommand{\hask}[1]{\mintinline{Haskell}{#1}}
\newenvironment{haskell}
  {\VerbatimEnvironment
  	\begin{minted}[escapeinside=??, mathescape=true,frame=single, framesep=5pt, tabsize=1]{Haskell}}
  {\end{minted}}
\newcommand{\cat}[1]{\mathcal{#1}}%a generic category
\newcommand{\Cat}[1]{\mathbf{#1}}%a named category
\newcommand{\Set}{\Cat{Set}}

\author{Bartosz Milewski}
\title{Presheaf Optics}
\begin{document}
\maketitle{}

\section{Motivation}


\section{Actegories}

An actegory is a category $\cat C$ with a (coherent) action of a monoidal category $\cat M$. Actegories for a given $\cat M$ form a 2-category. Here, I'd like to define a larger category that combines actegories for different monoidal categories.

We start with two monoidal categories $\cat M$ and $\cat N$ with two separate actions on two categories:
\begin{align*} 
\bullet &\colon \cat M \times \cat C \to \cat C 
\\
 \bullet &\colon \cat N \times \cat D \to \cat D 
\end{align*}
The two monoidal actions can also be seen, after currying, as strict monoidal functors:
\begin{align*}
&\cat M \to [\cat C, \cat C]
\\
&\cat N \to [\cat D, \cat D]
\end{align*}

Next we consider a category of monoidal functors $\cat M \to \cat N$. Given a functor $c$, we'll write its action on an object $m \colon \cat M$ as $c \cdot m$. It is given in terms of composition of the underlying functors. This action produces an object of $\cat N$. It satisfies the usual associativity and unit conditions of a monoidal functor. 

We also demand that there be a left adjoint to this functor, defining the action of $c$ on $\cat N$:

\[ \cat N (n, c \cdot m) \cong \cat M (c^{\dagger} n, m) \]

We lift the action of $c$ to a functor from $\cat C$ to  $\cat D$, or the action:
\[ \bullet \colon \cat M \times \cat C \to \cat D \]
satisfying the coherence conditions:
\[
 \begin{tikzcd}
 a
 \arrow [r, "m \bullet"] 
 \arrow [d, "(c \cdot m) \bullet"']
 & m \bullet a
 \arrow [d, "c \bullet"]
 \\
 (c \cdot m) \bullet a
 \arrow [r, "\cong"]
 & c \bullet (m \bullet a)
 \end{tikzcd}
\]

\[
 \begin{tikzcd}
 a
 \arrow [r, "c \bullet"] 
 \arrow [d, "(n \cdot c) \bullet"']
 & c \bullet a
 \arrow [d, "n \bullet"]
 \\
 (n \cdot c) \bullet a
 \arrow [r, "\cong"]
 & n \bullet (c \bullet a)
 \end{tikzcd}
\]
where $n \cdot c$ is defined as $c^{\dagger} n$.

We demand that there be an adjoint action:

\[ \cat D (b, c \bullet a) \cong \cat C (c^{\dagger} \bullet b, a) \]

\section{Appendix}

\[P \colon \cat D^{op} \times \cat C \to \Set \]
\[ \Psi \colon [\cat D, \Set] \to [\cat C, \Set] \]


\[ F \colon [\cat D, \Set] \]
$\Psi$ is a left adjoint so it preserves colimits.
\[P \langle d, c \rangle = \Psi ( \cat D(d, -)) c \]


\[ \Psi (F) c = \int^{d'} F d' \times P \langle d', c \rangle \]

Show that these are the inverse of each other. Inserting the formula for $\Psi$ in the definition of $P$ should gives us $P$ back:
\[  \int^{d'} \cat D(d, -) d' \times P\langle d', c \rangle \cong  P \langle d, c \rangle \]
which follows from the co-Yoneda lemma.

Inserting the formula for $P$ into the definition of $\Psi$ should give us $\Psi$ back:
\[  \int^{d'} F d' \times \Psi(\cat D(d', -)) c  \cong \Psi (F) c  \]


To prove this, we check it first on representables of the form:
\[ F d = \cat D (r, d) \]
We want to show that:
\[ \int^{d'}  \cat D (r, d')  \times  \Psi(\cat D(d', -)) c \cong  \Psi ( \cat D (r, -) ) c\]
This follows from the co-Yoneda lemma.

Every co-presheaf is a colimit of representables. Since $\Psi$ is a left adjoint, it preserves all colimits, so the formula works for any co-presheaf.


\end{document}